\section{Conclusion}

We have introduced the largest open dataset of high resolution images with polygon segmentations of dog
  poop.
While only focused on a single class, it is prototypical of challenges that arise in small-waste detection
  relevant to waste monitoring, pollution tracking, and environmental surveillance.
The dataset includes amorphous objects, occlusion, multi-season variation, difficult distractors, daytime /
  nighttime variation.
We have described the dataset collection and annotation process and reported statistics on the dataset.

We provided a recommended train/validation/test split of the dataset, and trained baseline segmentation
  models that perform well, but could likely be improved.
In addition to providing quantitative and qualitative results of the models, we also estimate the resources
  required to perform these training, prediction, and evaluation experiments.

We have published our data and models under a permissive license, and made them available through both
  centralized (Girder and HuggingFace) and decentralized (BitTorrent and IPFS) mechanisms.
Decentralized methods are robust, but suffer from significant network transfer overhead.
HuggingFace has exceptionally fast transfer speeds, has some decentralized
  properties, but lacks content identifiers.
%and due to its usage of git-lfs has some decentralized
%Combining IPFS with a content distribution network may offer promising directions.
%It may be possible to build a best of both worlds protocol and distribution network.

%Limitations of our work include:
%1) geographic concentration of the dataset,
%2) the small size of the independent test set,
%3) limited exploration of the better-performing model variant, and
%4) uncontrolled network conditions during distribution experiments.
%Future work could address these by expanding dataset diversity, training a
%broader range of models, and improving decentralized hosting strategies.

Our dataset enables applications such as mobile feces detection, urban cleanliness monitoring, and
  augmented-reality collision warnings.
Because it trains models to recognize small, irregular, low-contrast objects in cluttered scenes, we predict
  that including ``ScatSpotter'' in foundational training corpora will improve robustness to camouflage and
  small-object ambiguity in a broad range of ecological and waste-monitoring downstream tasks.


%Our dataset enables applications such as mobile apps for detecting feces, urban
%cleanliness monitoring, and augmented reality collision warnings. We believe
%negative impacts are limited and expect respectful use of the dataset.
%We envision exciting possibilities for the BAN protocol in computer vision research.
%We hope our work will inspire others to consider decentralized content addressable data sharing, fostering
%  open collaboration and reproducible experiments.
%Furthermore, we encourage the community to track experimental resource usage to better understand and offset
%  our experiments' small, but real environmental impact.
%Moreover, we aspire for our dataset to enable the creation of poop-aware applications.
%Ultimately, our goal is for this research to contribute meaningfully to the advancement of computer vision
%  and have a positive impact on society.
  
\FloatBarrier
  
  
%\ifnonanonymous
\ifuseacknowledgement
\section{Acknowledgements}
We would like to thank all of the dogs that produced subject matter for the dataset, all of the contributors
  for helping to construct a challenging test set, and \redact{Anthony Hoogs} for several suggestions
  including taking the third negative picture.
This work is dedicated to \redact{Bezoar}, a very weird and very good girl; \redact{Honey}, the sweetest
  red-fox lookalike; and \redact{Roadie}, a vicious cuddlebug with a deceptively soft face.

%\fi
\fi


\begin{comment}
We have introduced the largest open dataset of high resolution images with polygon segmentations of dog poop.
This dataset captures a challenging visual domain characterized by small amorphous objects, multi-season variation, and
distracting background clutter.  We have described the dataset collection and
annotation process, including the before/after/negative (BAN) protocol, and
reported statistics on the dataset.

We provided a recommended train/validation/test split and established baseline performance using several
detection and segmentation models. These models perform reasonably well but leave meaningful
room for improvement. We also quantified the computational resources required for training, prediction, and
evaluation to support reproducibility and planning for future work.

To facilitate open access, we released the dataset and models under permissive licenses and compared
centralized (Girder, HuggingFace) and decentralized (BitTorrent, IPFS) distribution mechanisms. 
Centralized systems offer convenience and high speed but rely on institutional hosting, while decentralized
systems provide robustness and content integrity at the cost of higher transfer overhead. Hybrid approaches,
such as integrating IPFS with content distribution networks, may offer promising directions.

Limitations of our work include geographic concentration of the dataset, a small independent test set,
limited exploration of top-performing model variants, and uncontrolled network conditions during distribution
experiments. Future work could expand geographic and species diversity, evaluate additional architectures,
and improve decentralized hosting strategies.

ScatSpotter supports research applications ranging from mobile feces detection to urban cleanliness
monitoring, augmented-reality warning systems, and broader small-object waste detection tasks. We hope this
dataset, its BAN protocol, and our exploration of distribution methods will encourage reproducible,
resource-aware experimentation and further progress in environmentally focused computer vision.
\end{comment}
