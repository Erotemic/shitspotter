% Note: Figures are in section1 for better late layout

%-------------------------------------------------------------------------
\section{Related Work}
\label{sec:relatedwork}

To the best of our knowledge, our dataset is currently the largest publicly available collection of
  annotated dog poop images, but it is not the first.
A dataset of 100 dog poop images was collected and used to train a FasterRCNN model
  \cite{neeraj_madan_dog_2019} but this dataset and model are not publicly available.
The company iRobot has a dataset of annotated indoor poop images used to train Roomba j7+ to avoid
  collisions \cite{roomba_2021}, but as far as we are aware, this is not available.
In terms of available poop detection datasets we are only aware of MSHIT~\cite{mshit_2020} which is much
  smaller, only contains box annotations, and the objects of interest are plastic toy poops.

Compared to benchmark object localization and segmentation datasets~\cite{ILSVRC15,
  lin_microsoft_2014,cordts2015cityscapes} ours is much smaller and focused only on a single category.
However, when compared to litter and trash datasets
  \cite{bashkirova_zerowaste_2022,proenca_taco_2020,hong2020trashcansemanticallysegmenteddatasetvisual,mittal2016spotgarbage,rs13050965}
  ours is among the largest in terms of number of images / annotations, image size, and total dataset size.
ZeroWaste~\cite{bashkirova_zerowaste_2022} uses a ``before/after'' protocol similar to our BAN protocol.
%% https://paperswithcode.com/dataset/tackknnno
We provide an overview of these related datasets in \Cref{tab:related_datasets}.
Among all of these, ours stands out for having the highest resolution images and the smallest objects
  relative to that resolution.
For a review of additional waste related datasets, refer to \cite{agnieszka_waste}.

\Cref{sec:dataset_transfer} discusses the logistics and tradeoffs between dataset distribution mechanisms
  with a focus on comparing centralized and decentralized methods.
IPFS~\cite{benet_ipfs_2014} and BitTorrent~\cite{cohen_incentives_2003} are the decentralized 
  mechanisms we evaluate, but others exist such as Secure Scuttlebut \cite{tarr_secure_2019} and Hypercore
  \cite{frazee_dep-0002_nodate}, which we did not test.

% Very good overview and comparison of the protocols
% https://blog.mauve.moe/posts/protocol-comparisons
% https://distributed.press/
% hypercore - https://github.com/tradle/why-hypercore/blob/master/FAQ.md#how-is-hypercore-different-from-ipfs
% git,
% Secure Scuttlebut (SSB)
