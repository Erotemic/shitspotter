\section{Introduction}
\label{sec:intro}

Applications for a computer vision system capable of detecting and localizing poop in images are numerous.
These include automated waste disposal to keep parks and backyards clean, tools for monitoring wildlife
  populations via droppings, and a warning system in smart-glasses to prevent people from stepping in poop.
Our primary motivating use case is a phone application that assists dog owners in locating their dog's poop
  in a leafy park for easier cleanup.
Many of these applications can be realized with modern object detection and segmentation methods
  \cite{sandler_mobilenetv2_2018, siam_rtseg_2018, yu_mobilenet_yolo_2023} combined with a large labeled
  dataset. 
%In this paper, we make a significant step towards building this dataset.


\begin{figure}[t]
\centering
\begin{subfigure}[t]{0.49\textwidth}
    \centering
    \includegraphics[width=\linewidth]{figures/zoom_leaf.jpg}
    \caption[]{
        A zoomed in example of an annotated object in a challenging
        condition: a scene cluttered with leaves. The similarity between the leaves
        and the poop causes a camouflage effect that can make detecting it difficult.
        The poop is highlighted in blue.
    }
    \label{fig:HardCase}
\end{subfigure}
\hfill
%\ifwacv
%\par\smallskip  % space between rows (optional)
%\fi
\begin{subfigure}[t]{0.49\textwidth}
    \centering
    \includegraphics[width=\linewidth]{figures/viz_three_images.jpg}
    \caption[]{
        The ``before/after/negative'' protocol.
        The orange box highlights the location of the poop 
        in the ``before'' image.
        In the ``after'' image, it is the same scene but the poop has been removed.
        The ``negative'' image is a nearby similar scene, potentially with a distractor.
        Note that the object is small relative to the image size.
    }
    \label{fig:ThreeImages}
\end{subfigure}
\caption{(a) A challenging annotation case due to camouflage. (b) The BAN protocol.}
\label{fig:Combined}
\end{figure}
  

\newcommand{\RelatedDatasetCaption}{
    \caption{Related datasets.
    %
    Columns list dataset name, number of categories, images, and annotations.
    Image W \times{} H gives median image dimensions;
    Ann Area$^{0.5}$ is the median square root of annotation area (pixels);
    Size is disk requirements in GB; 
    Annot Type is the labeling method.
    \Cref{fig:compare_allannots} shows the distribution of annotation shapes, sizes, and locations.
    %Of the datasets in this table, ours has the highest image resolution.
    %and the smallest annotation size relative to that resolution.
    %Of the waste related datasets and in terms of number of images, ours is among the largest, and of the poop related datasets, it is the largest.
    }
    \label{tab:related_datasets}
}
\begin{table*}[t]
\centering
\ifwacv \else \RelatedDatasetCaption \fi
\begin{tabular}{lrrrcrrl}
\toprule
Name & \#Cats & \#Images & \#Annots & \makecell{Image\\W \times{} H} & \makecell{Annot\\Area$^{0.5}$} & \makecell{Disk\\Size} & \makecell{Annot\\Type} \\
\midrule
ImageNet\cite{ILSVRC15}    & 1,000 & 594,546 & 695,776 & 500 \times{} 374 & 239 & 166GB & box \\
MSCOCO\cite{lin_microsoft_2014}      & 80 & 123,287 & 896,782 & 428 \times{} 640 & 57 & 50GB & polygon \\
CityScapes\cite{cordts2015cityscapes}  & 40 & 5,000 & 287,465 & 2,048 \times{} 1,024 & 50 & 78GB & polygon \\
ZeroWaste \cite{bashkirova_zerowaste_2022}   & 4 & 4,503 & 26,766 & 1,920 \times{} 1,080 & 200 & 10GB & polygon \\
TrashCanV1\cite{hong2020trashcansemanticallysegmenteddatasetvisual}  & 22 & 7,212 & 12,128 & 480 \times{} 270 & 54 & 0.61GB & polygon \\
UAVVaste\cite{rs13050965}    & 1 & 772 & 3,718 & 3,840 \times{} 2,160 & 55 & 2.9GB & polygon \\
SpotGarbage\cite{mittal2016spotgarbage} & 1 & 2,512 & 337 & 754 \times{} 754 & 355 & 1.5GB & category \\
TACO\cite{proenca_taco_2020}        & 60 & 1,500 & 4,784 & 2,448 \times{} 3,264 & 119 & 17GB & polygon \\
MSHIT\cite{mshit_2020}       & 2 & 769 & 2,348 & 960 \times{} 540 & 99 & 4GB & box \\
Ours        & 1 & 9,296 & 6,594 & 4,032 \times{} 3,024 & 87 & 60GB & polygon \\
\bottomrule
\end{tabular}
\ifwacv \RelatedDatasetCaption \fi
\end{table*}

\begin{figure*}[t]
\centering
\includegraphics[width=1.0\textwidth]{plots/appendix/dataset_compare/combo_all_polygons.png.png}
\caption[]{
    A comparison of all of the annotations for different datasets including ours.
    All polygon annotations drawn in a single plot with $0.8$ opacity to
    demonstrate the distribution in annotation location, shape, and size with
    respect to image coordinates.
}
\label{fig:compare_allannots}
\end{figure*}

\begin{figure*}[t]
\centering
\includegraphics[width=1\textwidth]{figures/umap-v3.jpg}%
\caption[]{
    %Example images from the dataset based on 2D UMAP \cite{mcinnes_umap_2020} clusters over the dataset.
    %Each point in the top image is a 2D-projected image embedding. Each
    %numbered orange dot corresponds to three nearby images, which are drawn in columns on the bottom.
    %Annotation boxes are drawn in blue.
    %An interesting observation is that there is a clear separation into two UMAP blobs represents snowy versus (columns 1 and 2)
    %  non-snowy images (columns 3-13). We verified that this pattern holds beyond the examples explicitly shown here.
    Example images from 2D UMAP clusters \cite{mcinnes_umap_2020}.
    Each point in the top image represents a 2D-projected embedding, with numbered orange dots indicating nearby
      images in the bottom columns.
    Blue annotation boxes are shown.
    A clear separation emerges between snowy (columns 1-2) and non-snowy images (columns 3-13).
    %this pattern verified beyond these examples.
    %a 200 image subset
    %  of the dataset.
    %Each row corresponds to a selection from a 2D UMAP projection shown on the left.
    %The highlighted nodes circled in blue in the cluster visualization in each row correspond to the images
    %  with annotations (drawn in green) shown on the right.
}
\label{fig:umap_dataset_viz}
\end{figure*}


In addition to enabling several applications, poop detection is an interesting benchmark problem.
It is relatively simple, with a narrow focus on a single class, making it suitable for exploring the
  capabilities of object detection models that target a single labeled class.
However, the task includes non-trivial challenges such as resolution issues (e.g., camera quality,
  distance), camouflaging distractors (e.g., leaves, pine cones, sticks, dirt, and mud), occlusion (e.g., bushes, overgrown
  grass), and variation in appearance (e.g., old vs. new, healthy vs. sick).
An example of a challenging case is shown in \Cref{fig:HardCase}.
Investigation into cases where this problem is difficult may provide insight
into how to better train object detection and segmentation networks.

Towards these ends we introduce a new dataset which, 
%for the purpose of this paper we call "ScatSpotter".
%we formally call ``ScatSpotter''.
in formal settings, we call ``ScatSpotter''.
Poops are annotated with polygons, making the dataset suitable for training detection and segmentation
  models.
In order to assist with annotation and add variation, we collect images using a ``before/after/negative'' (BAN)
  protocol as shown in \Cref{fig:ThreeImages}.

From this data, we train a segmentation model to classify which pixels in an image contain poop and which do
  not.
Our models show strong performance, but there are notable failure cases indicating this problem is difficult
  even for modern computer vision algorithms. 

To enable others to build on our results, it is essential that the dataset is accessible and hosted
  reliably.
Centralized methods are a typical choice, offering high speeds, but they can be costly for individuals,
  often requiring institutional support or paid hosting services.
They are also prone to outages and lack built-in data validation.
In contrast, decentralized methods allow volunteers to host data and offers built-in validation of data
  integrity.
This motivates us to compare and contrast the decentralized BitTorrent \cite{cohen_incentives_2003}, and
  IPFS \cite{benet_ipfs_2014} protocols as mechanisms for distributing datasets.

% VGG2 face got removed.
% https://github.com/ox-vgg/vgg_face2/issues/52

Our contributions are:
1) A challenging new \textbf{open dataset} of images with polygon annotations.
2) A set of trained \textbf{baseline models}.
3) A \textbf{comparison of dataset distribution} methods.
%4) \textbf{Open code and models}.
