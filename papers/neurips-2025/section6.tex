\section{Conclusion}

We have introduced the largest open dataset of high resolution images with polygon
  segmentations of dog poop.
The dataset contains several challenges including amorphous objects, multi-season variation, difficult
  distractors, daytime / nighttime variation.
We have described the dataset collection and annotation process and reported statistics on the dataset.

We provided a recommended train/validation/test split of the dataset, and trained baseline segmentation
  models that perform well, but could likely be improved.
In addition to providing quantitative and qualitative results of the models, we also estimate the resources
  required to perform these training, prediction, and evaluation experiments.

We have published our data and models under a permissive license, and made them available through both
  centralized (Girder and HuggingFace) and decentralized (BitTorrent and IPFS) mechanisms.
Decentralized methods have robustness properties, but suffer from significant network transfer overhead.
HuggingFace has exceptionally fast transfer speeds, and due to its usage of git-lfs has some decentralized
  properties, but lacks content identifiers.
Combining IPFS with a content distribution network may be a path to a best-of-both-worlds system.
%It may be possible to build a best of both worlds protocol and distribution network.


Limitations of our work include:
1) geographic concentration of the dataset,
2) the small size of the independent test set,
3) limited exploration of the better-performing model variant, and
4) uncontrolled network conditions during distribution experiments.
Future work could address these by expanding dataset diversity, training a
broader range of models, and improving decentralized hosting strategies.

Our dataset enables applications such as mobile apps for detecting feces, urban
cleanliness monitoring, and augmented reality collision warnings. We believe
negative impacts are limited and expect respectful use of the dataset.
We envision exciting possibilities for the BAN protocol in computer vision research.
We hope our work will inspire others to consider decentralized content addressable data sharing, fostering
  open collaboration and reproducible experiments.
Furthermore, we encourage the community to track experimental resource usage to better understand and offset
  our experiments' small, but real environmental impact.
Moreover, we aspire for our dataset to enable the creation of poop-aware applications.
Ultimately, our goal is for this research to contribute meaningfully to the advancement of computer vision
  and have a positive impact on society.
  
\ifwacv
\else
\FloatBarrier
\fi
  
%\ifnonanonymous
\ifuseacknowledgement
\section{Acknowledgements}
We would like to thank all of the dogs that produced subject matter for the dataset, all of the
contributors for helping to construct a challenging test set, and \redact{Anthony Hoogs} for several suggestions including taking the 
  third negative picture.
This work is dedicated to \redact{Bezoar}, a very weird and very good girl.

%\fi
\fi
