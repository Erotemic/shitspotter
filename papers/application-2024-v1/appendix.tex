

\appendix

\section{Extra Dataset Information}

In \Cref{sec:dataset} we provided an overview of several dataset statistics.
In this appendix we expand on that with additional plots.
The distribution of image pixel intensities is illustrated in \Cref{fig:spectra}.
The distribution of images collected over time is shown in \Cref{fig:images_over_time}.
The distribution of annotation location is shown in \Cref{fig:centroid_location_distri} and sizes is shown
  in \Cref{fig:annot_obox_size_dist} and \Cref{fig:annot_area_verts_distri}.


\begin{figure}[ht]
\centering
\includegraphics[width=0.4\textwidth]{figures/spectra.png}
\caption[]{
    The "spectra" or histogram of the pixel intensities in the dataset. 
    The dataset rgb  mean/std is $[117, 124, 100], [61, 59, 63]$.
    % (for reference the imagenet mean/std is $[124, 116, 104], [58, 57, 57]$).
}
\label{fig:spectra}
\end{figure}


\begin{figure}[ht]
\centering
\includegraphics[width=0.4\textwidth]{figures/appendix/images_over_time.png}
\caption[]{
    The number of images collected over time.
}
\label{fig:images_over_time}
\end{figure}


\begin{figure}[ht]
\centering
\includegraphics[width=0.4\textwidth]{figures/appendix/polygon_centroid_absolute_distribution.png}

(a) absolute pixel coordinates
\includegraphics[width=0.4\textwidth]{figures/appendix/polygon_centroid_relative_distribution.png}

(b) relative image coordinates
\caption[]{
    The distribution of annotation centroids in terms of (a) absolute image
    coordinates and (b) relative image coordinates.
    The absolute centroid distribution is bimodal because some images are taken
    in landscape mode and other in portrait mode.
}
\label{fig:centroid_location_distri}
\end{figure}


\begin{figure}[ht]
\centering
\includegraphics[width=0.4\textwidth]{figures/appendix/obox_size_distribution_jointplot.png}%

(a) linear scale
\includegraphics[width=0.4\textwidth]{figures/appendix/obox_size_distribution_logscale.png}%

(b) log10 scale
\caption[]{
    The distribution of annotation sizes as measured by an oriented bounding box fit to each polygon.
    (a) shows this plot on a linear scale and (b) show this plot on a log scale.
}
\label{fig:annot_obox_size_dist}
\end{figure}


\begin{figure}[ht]
\centering
\includegraphics[width=0.4\textwidth]{figures/appendix/polygon_area_vs_num_verts_jointplot.png}
\caption[]{
    The distribution of polygon areas versus the number of vertices in the polygon boundary.
    The SAM model tends to produce polygons with a higher number of vertices
    than manually drawn ones.  For smaller polygons there are two peaks in the
    number of vertices histograms likely corresponding to pure-manual versus
    AI-assisted annotations.
}
\label{fig:annot_area_verts_distri}
\end{figure}

\section{Extra Experiment Information}

In \Cref{sec:models} we presented an experimental evaluation of several models trained with different
  hyperparameters.
This involved evaluating several checkpoints produced by each model run.
Previously in \Cref{tab:parameters_and_results} we presented the top results.
Here we've plotted the AP and AUC on the validation set for the top 5 AP-maximizing results from each of the
  7 training runs.
We also created a box-and-whisker plot for these top 5 results, which serves to assign a color and label to
  each training run.
These plots are shown in \Cref{fig:apauc_scatter}.


\begin{figure}[ht]
\centering

\includegraphics[width=0.4\textwidth]{figures/macro_results_resolved_params.heatmap_pred_fit.trainer.default_root_dir_metrics.heatmap_eval.salient_AP_vs_metrics.heatmap_eval.salient_AUC_PLT02_scatter_nolegend.png}

(a) AP and AUC of 35 checkpoints.

\includegraphics[width=0.4\textwidth]{figures/macro_results_resolved_params.heatmap_pred_fit.trainer.default_root_dir_metrics.heatmap_eval.salient_AP_PLT04_box.png}

(b) AP of 35 checkpoints.

\caption[]{
    (a) Scatterplot of pixelwise average precision (AP) and Area Under the ROC curve (AUC) for the top
      5 checkpoints on the validation set.
    Points of the same color represent checkpoints from the same training run, which used identical
      hyperparameters.
    (b) Box-and-whisker plot the AP values across the top 5 checkpoints evaluated on
      the validation set.
    For each run, corresponding varied hyperparameters and maximum APs are given in
      \Cref{tab:parameters_and_results}.
}
\label{fig:apauc_scatter}
\end{figure}


\section{Extra Dataset Comparison}

In \Cref{sec:relatedwork} we compared to related work. Here we expand on this
by comparing our analysis plots. Every dataset is converted into the COCO
format and visualized using the same logic. \Cref{fig:compare_allannots}
visualizes the annotations of all datasets. We make similar visualizations 
for other comparable dataset metrics.
\Cref{fig:combo_anns_per_image_histogram_splity} shows the number of annotations per image.
\Cref{fig:combo_image_size_scatter} shows of image sizes in each dataset.
\Cref{fig:combo_obox_size_distribution_logscale} shows the distribution of width and heights of oriented bounding boxes fit to annotation polygons.
\Cref{fig:combo_polygon_area_vs_num_verts_jointplot} shows the area of each polygon versus the number of vertices (which could be used to estimate the likelihood a polygon was generated by AI for our dataset).
\Cref{fig:combo_polygon_centroid_relative_distribution} show the distribution of centroid positions (relative to the image size).


\begin{figure*}[ht]
\centering
\includegraphics[width=1.0\textwidth]{plots/appendix/dataset_compare/combo_all_polygons.png.png}
\caption[]{
    A comparison of all of the annotations for different datasets including ours.
    All polygon annotations drawn in a single plot with 0.8 opacity to
    demonstrate the distribution in annotation location, shape, and size with
    respect to image coordinates.
}
\label{fig:compare_allannots}
\end{figure*}


\begin{figure*}[ht]
\centering
\includegraphics[width=1.0\textwidth]{plots/appendix/dataset_compare/combo_anns_per_image_histogram_splity.png.png}
\caption[]{
    Number of annotations per image in each dataset.
}
\label{fig:combo_anns_per_image_histogram_splity}
\end{figure*}


\begin{figure*}[ht]
\centering
\includegraphics[width=1.0\textwidth]{plots/appendix/dataset_compare/combo_image_size_scatter.png.png}
\caption[]{
    Image size distributions of each dataset.
}
\label{fig:combo_image_size_scatter}
\end{figure*}


\begin{figure*}[ht]
\centering
\includegraphics[width=1.0\textwidth]{plots/appendix/dataset_compare/combo_obox_size_distribution_logscale.png.png}
\caption[]{
    Oriented bounding box size distributions (log10 scale) of each dataset.
}
\label{fig:combo_obox_size_distribution_logscale}
\end{figure*}

\begin{figure*}[ht]
\centering
\includegraphics[width=1.0\textwidth]{plots/appendix/dataset_compare/combo_polygon_area_vs_num_verts_jointplot_logscale.png.png}
\caption[]{
    Polygon area versus number of vertices (log10 scale) for each dataset.
}
\label{fig:combo_polygon_area_vs_num_verts_jointplot}
\end{figure*}

\begin{figure*}[ht]
\centering
\includegraphics[width=1.0\textwidth]{plots/appendix/dataset_compare/combo_polygon_centroid_relative_distribution.png.png}
%(a) scatter and kde plot
%\includegraphics[width=1.0\textwidth]{plots/appendix/dataset_compare/combo_polygon_centroid_relative_distribution_jointplot.png.png}
%(b) jointplot
\caption[]{
    Polygon centroid relative distribution for each dataset.
}
\label{fig:combo_polygon_centroid_relative_distribution}
\end{figure*}


