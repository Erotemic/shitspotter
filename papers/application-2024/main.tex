% WACV 2024 Paper Template
% based on the CVPR 2023 template (https://media.icml.cc/Conferences/CVPR2023/cvpr2023-author_kit-v1_1-1.zip) with 2-track changes from the WACV 2023 template (https://github.com/wacv-pcs/WACV-2023-Author-Kit)
% based on the CVPR template provided by Ming-Ming Cheng (https://github.com/MCG-NKU/CVPR_Template)
% modified and extended by Stefan Roth (stefan.roth@NOSPAMtu-darmstadt.de)

\documentclass[10pt,twocolumn,letterpaper]{article}

%%%%%%%%% PAPER TYPE  - PLEASE UPDATE FOR FINAL VERSION
%\usepackage[review,algorithms]{wacv}      % To produce the REVIEW version for the algorithms track
%\usepackage[review,applications]{wacv}      % To produce the REVIEW version for the applications track
\usepackage{wacv}              % To produce the CAMERA-READY version
%\usepackage[pagenumbers]{wacv} % To force page numbers, e.g. for an arXiv version

% Include other packages here, before hyperref.
\usepackage{graphicx}
\usepackage{amsmath}
\usepackage{amssymb}
\usepackage{booktabs}


% It is strongly recommended to use hyperref, especially for the review version.
% hyperref with option pagebackref eases the reviewers' job.
% Please disable hyperref *only* if you encounter grave issues, e.g. with the
% file validation for the camera-ready version.
%
% If you comment hyperref and then uncomment it, you should delete
% ReviewTempalte.aux before re-running LaTeX.
% (Or just hit 'q' on the first LaTeX run, let it finish, and you
%  should be clear).
\usepackage[pagebackref,breaklinks,colorlinks]{hyperref}


% Support for easy cross-referencing
\usepackage[capitalize]{cleveref}
\crefname{section}{Sec.}{Secs.}
\Crefname{section}{Section}{Sections}
\Crefname{table}{Table}{Tables}
\crefname{table}{Tab.}{Tabs.}


%%%%%%%%% PAPER ID  - PLEASE UPDATE
\def\wacvPaperID{*****} % *** Enter the WACV Paper ID here
\def\confName{WACV}
\def\confYear{2024}


\begin{document}

\title{ShitSpotter --- A Dog Poop Detection Algorithm and Dataset}

\author{Jonathan Crall\\
Kitware\\
{\tt\small jon.crall@kitware.com}
%{\tt\small erotemic@gmail.com}
% For a paper whose authors are all at the same institution,
% omit the following lines up until the closing ``}''.
% Additional authors and addresses can be added with ``\and'',
% just like the second author.
% To save space, use either the email address or home page, not both
%\and
%Second Author\\
%Institution2\\
%First line of institution2 address\\
%{\tt\small secondauthor@i2.org}
}
\maketitle

%%%%%%%%% ABSTRACT
\begin{abstract}

    This work chronicles one researcher's un-funded journey to build a phone
    application that can detect dog poop in images, and make the data widely
    available as a benchmark dataset.


\end{abstract}

%%%%%%%%% BODY TEXT
\section{Introduction}
\label{sec:intro}


Poop detection is a simple problem, suitable for exploring the capabilities of
object detection models while also containing non-trivial challenges.

There are several challenges in detecting dog poop in phone-camera images.
* Resolution
* Distractors
* Occlusion
* Variation in appearance (old/new/healthy/sick)


Hosting a dataset challenges intended for scientific use
* Requires an institution willing to host or payment for a hosting service
* Prone to outages (give VGG outage as example)
* Requires updates (makes bittorrent difficult)


Compare and Contrast:
* Centralized
* BitTorrent
* IPFS

% https://gist.github.com/liamzebedee/4be7d3a551c6cddb24a279c4621db74c
% https://gist.github.com/liamzebedee/224494052fb6037d07a4293ceca9d6e7


% https://arxiv.org/abs/1803.09010
Data is released with a datasheet describing its characteristics \cite{gebru_datasheets_2021}.

% BitTorrent can be vulnerable to MITM:
% https://www.reddit.com/r/technology/comments/1dpinuw/south_korean_telecom_company_attacks_torrent/


%-------------------------------------------------------------------------
\subsection{Related Work}

Object detection

TACO dataset: \cite{proenca_taco_2020}

MSHIT dataset <cite>

Dog Poop Detection - Neeraj Madan

Other poop work

\subsection{Dataset Construction}

Labelme \cite{wada_labelmeailabelme_nodate} for annotations with segment anything \cite{kirillov_segment_2023}.

Anecdotal note: SAM worked well to automatically segment the poop, many of
these needed adjustments, especially in regions of shadows, but there were
cases that required a completely manual approach. Unfortunately a clean record
of what cases these were does not exist. 

\subsection{Dataset Distribution}

Discuss distributing the dataset via IPFS versus centralized distribution
systems.

Decentralized Method - IPFS
Centralized Method - Girder

Observations:
* IPFS via https using gateways does not always work well.
* IPFS usually works well if you use the CLI.
* IPFS is easier to update.
* IPFS does rehash every file, which induces an O(N) scalability constraint.
* IPFS does rehash every file, which induces an O(N) scalability constraint.


IPFS vs BitTorrent:
https://gist.github.com/liamzebedee/224494052fb6037d07a4293ceca9d6e7

Kademlia - distributed hash table [Steiner, En-Najjary, Biersack 2022]

The Mainline Tracker is a DHT for bittorrent.

% See Also:
% Long Term Study of Peer Behavior in the KAD DHT
% https://git.gnunet.org/bibliography.git/plain/docs/Long_Term_Study_of_Peer_Behavior_in_the_kad_DHT.pdf
% We have been crawling the entire KAD network once a day for more than a year to track end-users with static
% IP addresses, which allows us to estimate end-user lifetime and the fraction of end-users changing their KAD ID.


Dataset is (will be) tracked on Academic Torrents \cite{academic_torrents_Cohen2014}.

https://academictorrents.com/docs/about.html


\subsection{Experiments}

Measure the performance of our algorithm versus a baseline.

Measure the speed of IPFS vs bittorrent.

Define Model - Separated Attention Transformer

Define training protocol

Define Parameter Search Space - Learning Rate / 

Define quality metrics - Pixel IoU

Plot Scatter Plots and Box Plots

Make any inferences


\subsection{Future}

The future of the project will:

* Add lightweight object-level head and test object detection metrics
* Optimize model architectures for mobile devices
* Launch phone application
* Improve distributed distribution mechanisms

\subsection{Conclusion}

IPFS is a promising solution for hosting scientific datasets, but does have pain points.
In contrast bittorrent can do X/Y/Z, but ...
Lastly there are centralized systems which ...

Our dataset is sufficient to train an object detection network to (level of
precision/recall).



%%%%%%%%% REFERENCES
{\small
\bibliographystyle{ieee_fullname}
\bibliography{egbib}
}

\end{document}
